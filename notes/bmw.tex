\date{\today}

%\documentclass[journal=jacsat,manuscript=communication,layout=twocolumn]{achemso}
%\documentclass[jctcce,letterpaper,twocolumn,floatfix,preprintnumbers,superscriptaddress]{revtex4}
% \documentclass[12pt,preprint,aps,prb]{revtex4}
% \documentclass[aps,preprint,showpacs,superscriptaddress,groupedaddress]{revtex4}  % for double-spaced preprint
\documentclass[aps,prl,twocolumn,superscriptaddress,groupedaddress]{revtex4}  % for review and submission
\usepackage{dcolumn,graphicx,amsmath,amssymb,algorithm,algpseudocode}
\usepackage{todonotes}
\usepackage{qcircuit}

\newcommand{\total}{\mathrm{d}}
\newcommand{\ud}{\mathrm{d}}
\newcommand{\erf}{\mathrm{erf}}
\newcommand{\erfc}{\mathrm{erfc}}
\newcommand{\diff}[2]{\frac{\ud {#1}}{\ud {#2}}}
\newcommand{\pdiff}[2]{\frac{\partial #1}{\partial #2}}

\begin{document}

\definecolor{brickred}{rgb}{.72,0,0} 

\title{
BMW Quantum Challenge: Optimizing the Production of Test Vehicles
}

\author{Robert M. Parrish}
\email{rob.parrish@qcware.com}
\affiliation{
QC Ware Corporation, Palo Alto, CA 94301, USA \\
\textbf{QC Ware Corporation Proprietary and Confidential}
}


\begin{abstract} 
Notes of the formulation and implementation of BMW test vehicle production
optimization problem.
\end{abstract}

\maketitle

\section{Formulation}


\subsection{Notes on Problem Input Refinement}

\subsubsection{Group Feature Collision}

Their consultant is just trying to be an asshole:

Group 40 (28 elements): \texttt{[245, 246, 247, 250, 251, 252, 253, 254, 255,
256, 266, 267, 268, 269, 270, 271, 272, 273, 274, 275, 276, 277, 278, 279, 280,
281, 282, 284]}

Group 41 (46 elements): \texttt{[245, 246, 247, 248, 249, 250, 251, 252, 253,
254, 255, 256, 257, 258, 259, 260, 261, 262, 263, 264, 267, 268, 269, 270, 271,
272, 273, 274, 275, 276, 277, 278, 279, 280, 281, 282, 283, 285, 286, 287, 288,
289, 290, 291, 292, 293]}

Union (48 elements): \texttt{[245, 246, 247, 248, 249, 250, 251, 252, 253, 254,
255, 256, 257, 258, 259, 260, 261, 262, 263, 264, 266, 267, 268, 269, 270, 271,
272, 273, 274, 275, 276, 277, 278, 279, 280, 281, 282, 283, 284, 285, 286, 287,
288, 289, 290, 291, 292, 293]}

Intersection (26 elements): \texttt{[245, 246, 247, 250, 251, 252, 253, 254,
255, 256, 267, 268, 269, 270, 271, 272, 273, 274, 275, 276, 277, 278, 279, 280,
281, 282]}

In Group 40 but not Group 41 (2 elements): \texttt{[266, 284]}

In Group 41 but not Group 40 (20 elements): \texttt{[248, 249, 257, 258, 259,
260, 261, 262, 263, 264, 283, 285, 286, 287, 288, 289, 290, 291, 292, 293]}

This is hugely vexing for efficient enumeration of group-feature-satisfying
vehicle candidates.

This can be overcome by (1) redefining Group 40 to be \texttt{[266, 284]} and
then (2) adding a new global (added for all types) rule to the build rules:

\texttt{ F266 | F284 => !F245 \& !F246 \& !F247 \& !F250 \& !F251 \& !F252 \& !F253 \&
!F254 \& !F255 \& !F256 \& !F267 \& !F268 \& !F269 \& !F270 \& !F271 \& !F272 \& !F273 \&
!F274 \& !F275 \& !F276 \& !F277 \& !F278 \& !F279 \& !F280 \& !F281 \& !F282}

If the group features are chosen randomly, uniformly, and independently, this
rule has a probability of $2/(1+2)$ to be activated (if 266 xor 284 are true).
The probability of the rule being violated is $\sim 26/(1+46) \sim 0.55$.
Therefore the joint probability of the rule being activated and failing is
$(2/3) * (26/47) \sim 0.37$. Note that this high success probability is somewhat
accidental, and is only due to the fact that the in-40-but-not-in-41 subset is
small relative to the intersection \emph{and} the in-41-but-not-in-40 subset is
large relative to the the intersection. In future, it is recommended that
collisions between feature groups be avoided at all costs in the formulation of
this problem, insofar as is possible.

\subsection{Notes on Problem Input Specialization}

\subsubsection{Notes on Problem Rules}

We consider certain classes of rules observed in the pilot problem inputs to be
universal, and denote such choices here:

\begin{enumerate}
\item Type allowed feature indices are nondegenerate and sorted during parsing
to be stored in ascending order.
\item Group feature indices are nondegenerate and sorted during parsing to be stored
in ascending order. 
\item The union of Group feature indices is nondegenerate after refinement.
\item Not all Types have all Groups active.
\item Not all Type features are contained in a Group.
\item Not all Group features are valid for an intersecting Type.
\end{enumerate}

\subsubsection{Notes on Problem Sizes}

\begin{enumerate}
\item The number of Types $n_{T}$ is 25.
\item The number of Features $n_{F}$ is 469. 
\end{enumerate}

\subsubsection{Notes on Problem Seeding}

\begin{enumerate}
\item A guess of $\vec 0$ does not satisfy build rules for any type. Between $2$
and $4$ (inclusive) build rules are violated, depending on the type. All
implications of these build rules are ``any on'' types. The cardinalities are 2:
3, 3: 15, 4: 6.
\item Valid cars of all types can be found by chasing the $\vec 0$ build rules
zero levels deep to fix the predicates of the violated rules. There are between
4 and 609 valid cars per type.
\item Valid Type 2, 13, 24 cars (the smallest $\vec 0$ rule violators with
$2\times$ rules each) can be found by chasing the $\vec 0$ build rules
violations one level deep to fix the implications of the violated rules. There
are 150, 175, and 175 valid cars, respectively.
\item Valid Type 5 cars (the smallest type) can be found efficiently by random
search.
\end{enumerate}


% \bibliography{jrncodes.bib,refs.bib}
% \bibliographystyle{aip}

\end{document}
