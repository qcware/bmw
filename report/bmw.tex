\date{\today}

%\documentclass[journal=jacsat,manuscript=communication,layout=twocolumn]{achemso}
%\documentclass[jctcce,letterpaper,twocolumn,floatfix,preprintnumbers,superscriptaddress]{revtex4}
% \documentclass[12pt,preprint,aps,prb]{revtex4}
% \documentclass[aps,preprint,showpacs,superscriptaddress,groupedaddress]{revtex4}  % for double-spaced preprint
\documentclass[aps,prl,twocolumn,superscriptaddress,groupedaddress]{revtex4}  % for review and submission
\usepackage{dcolumn,graphicx,amsmath,amssymb,algorithm,algpseudocode}
\usepackage{todonotes}
\usepackage{qcircuit}

\newcommand{\total}{\mathrm{d}}
\newcommand{\ud}{\mathrm{d}}
\newcommand{\erf}{\mathrm{erf}}
\newcommand{\erfc}{\mathrm{erfc}}
\newcommand{\diff}[2]{\frac{\ud {#1}}{\ud {#2}}}
\newcommand{\pdiff}[2]{\frac{\partial #1}{\partial #2}}

\begin{document}

\definecolor{brickred}{rgb}{.72,0,0} 

\title{
BMW Quantum Challenge: Optimizing the Production of Test Vehicles
}

\author{Robert M. Parrish}
\email{rob.parrish@qcware.com}
\author{Rachael Al-Saadon}
\affiliation{
QC Ware Corporation, Palo Alto, CA 94301, USA \\
\textbf{QC Ware Corporation Proprietary and Confidential}
}


\begin{abstract} 
A complete and completely classical solution of the industrial challenge problem
is presented.  Additional gains are potentially possible for extended versions
of this problem and/or for similar problems if hybrid quantum/classical
algorithms are considered - we present some ideas along these lines.

% We solved your problem. \\
% Want to know how? \\
% We can do this with quantum too...\\
% Despite apparent classical complexity, this problem is, at present, not suitable for
% the possibility of practical quantum advantage. We demonstrate this pragmatically
% by direct and tight classical solution of the optimization problem as specified
% in the quantum computing challenge (while noting that this problem does not seem
% to exhibit the typical drastic simplification of ``quantum'' formulations of
% standard binary optimization problems throughout the industry). 
\end{abstract}

\maketitle

\section{Results}

The problem statement(s) variously ask for optimization of the constituents of a
set or
``constellation'' of $n_{\mathrm{C}}$ test vehicles, with each test vehicle taken
from a state space of $\sim 459$ binary dimensions (this and other dimensions
quoted below to vary
in future problem sizes), and with each test vehicle 
satisfying hard ``feature-group'' and ``type-build rule'' constraints
corresponding to $\sim 25$ basic test vehicle types. The problem
statement(s), predicated by the hard constraints, specifically ask for
(1) \textbf{SAT:} 
For a given $n_{\mathrm{C}}$, does there exist, for a given set of
$n_{\mathrm{test}} \sim 644$ tests depending through binary expressions on the state
space of each test vehicle, a set of $n_{\mathrm{C}}$ test cars for which the
$n_{\mathrm{test}}$ tests can be separately evaluated, with the caveat 
that there need be $K_I \sim 1-5$ distinct
test vehicles required to satisfy test $I$ for $I \in [0, n_{\mathrm{test}})$?
(2) \textbf{Weighted MAX-SAT:} For a given $n_{\mathrm{C}}$, what is the optimal
constellation of test vehicles such that the weighted sum of satisfied $n_{\mathrm{test}}$
tests, each requiring $K_I$ distinct test vehicles, is maximized? and 
(3) \textbf{Scheduling (not precisely specified):} For a given set of
$n_{\mathrm{test}}$ tests and corresponding set of $n_{\mathrm{C}}$ test
vehicles satisfying said tests including $\{ K_I \}$ multiplicity constraints in
a MAX-SAT formalism of (2), what is the optimal scheduling of said vehicles into 
a test sequence with at most $n_{\mathrm{slot}} \sim 10$ tests performed on
distinct cars in each timeslot and with tests assigned to integer test groups
with definite sorting of test groups within each car?

Within the problem statement document, solutions to the above problems were
attempted using existing industry-standard sat solvers and constraint
satisfaction solvers. The SAT problem of (1) was easily solved:
``For 100 cars, the problem can be solved in a few seconds. A linear search counting down
from 100 revealed the solution that at least 60 cars are needed to perform all the specified
750 tests.''
However the weighted MAX-SAT problem of (2) was not solvable:
``On the other hand, the MAX-SAT problem was not solvable in a reasonable time with the
chosen approach.
Additionally, the scheduling problem of (3) was not solvable:
``[O]n the test laptop, the full problem with 700 tests wasn't solvable in less than 24
hours.''

\textbf{We provide what we believe under the rules of the problem statement
represents a complete and tangible solution to all three specified problems.}
Specifically, we developed a custom C++/Python code library to represent the
details of the problem in a natural format. 
The combination of customized classical solution environment and high
performance implementation allows for very rapid exploration of the
hard-constraint-satisfying parameter space unique to this problem class.  Within
this environment, we developed a powerful and simple set of heuristics to
approximately solve the MAX-SAT variant of the problem. This heuristic MAX-SAT
solver produces nested constellations of test cars with increasing
$n_{\mathrm{C}}$ and concomitant increasing MAX-SAT scores. The MAX-SAT
solutions coming from this heuristic achieve saturation of all specified $644$
tests (including multiplicity considerations) at the same $n_{\mathrm{C}} = 60$
bound determined by standard SAT solvers for problem (1) in the problem
statement document. Thus our MAX-SAT solution provides a tight bound solution
for problem (1) in the process of providing approximate solutions for (2). For
values of $n_{\mathrm{C}} \ll 60$, we believe our heuristic MAX-SAT solutions
are within a few percent of the global optimum. For the scheduling problem of
(3) we develop additional heuristics to schedule the test sequence from the
MAX-SAT optimized constellation fo $n_{\mathrm{C}} = 60$ cars while respecting
the hard constraints of distinct cars within each time slot, strict ordering of
randomly-specified test groups within cars, and separate cars used within the
multiplicity considerations of each test. With the multiplicity considerations
included, there are 766 separate test-car pairs required, mandating a
theoretical floor of 77 test slots. Our heuristic solution provides a nearly
dense scheduling with 78 test slots required, i.e., within 1.3$\%$ of dense
scheduling.


\section{Formulation}

\subsubsection{Initial Guess}






% \bibliography{jrncodes.bib,refs.bib}
% \bibliographystyle{aip}

\end{document}
